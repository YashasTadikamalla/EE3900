\documentclass[journal,12pt,twocolumn]{IEEEtran}

\usepackage{setspace}
\usepackage{gensymb}
\singlespacing
\usepackage[cmex10]{amsmath}

\usepackage{amsthm}

\usepackage{mathrsfs}
\usepackage{txfonts}
\usepackage{stfloats}
\usepackage{bm}
\usepackage{cite}
\usepackage{cases}
\usepackage{subfig}

\usepackage{longtable}
\usepackage{multirow}

\usepackage{enumitem}
\usepackage{mathtools}
\usepackage{steinmetz}
\usepackage{tikz}
\usepackage{circuitikz}
\usepackage{verbatim}
\usepackage{tfrupee}
\usepackage[breaklinks=true]{hyperref}
\usepackage{graphicx}
\usepackage{tkz-euclide}

\usetikzlibrary{calc,math}
\usepackage{listings}
    \usepackage{color}                                            %%
    \usepackage{array}                                            %%
    \usepackage{longtable}                                        %%
    \usepackage{calc}                                             %%
    \usepackage{multirow}                                         %%
    \usepackage{hhline}                                           %%
    \usepackage{ifthen}                                           %%
    \usepackage{lscape}     
\usepackage{multicol}
\usepackage{chngcntr}

\DeclareMathOperator*{\Res}{Res}
\newtheorem{theorem}{Theorem}[section]
\newtheorem{corollary}{Corollary}[theorem]
\newtheorem{lemma}[theorem]{Lemma}
\newtheorem{definition}{Definition}[section]
\renewcommand\thesection{\arabic{section}}
\renewcommand\thesubsection{\thesection.\arabic{subsection}}
\renewcommand\thesubsubsection{\thesubsection.\arabic{subsubsection}}

\renewcommand\thesectiondis{\arabic{section}}
\renewcommand\thesubsectiondis{\thesectiondis.\arabic{subsection}}
\renewcommand\thesubsubsectiondis{\thesubsectiondis.\arabic{subsubsection}}


\hyphenation{op-tical net-works semi-conduc-tor}
\def\inputGnumericTable{}                                 %%

\lstset{
%language=C,
frame=single, 
breaklines=true,
columns=fullflexible
}
\begin{document}

\newcommand{\BEQA}{\begin{eqnarray}}
\newcommand{\EEQA}{\end{eqnarray}}
\newcommand{\define}{\stackrel{\triangle}{=}}
\bibliographystyle{IEEEtran}
\raggedbottom
\setlength{\parindent}{0pt}
\providecommand{\mbf}{\mathbf}
\providecommand{\pr}[1]{\ensuremath{\Pr\left(#1\right)}}
\providecommand{\qfunc}[1]{\ensuremath{Q\left(#1\right)}}
\providecommand{\sbrak}[1]{\ensuremath{{}\left[#1\right]}}
\providecommand{\lsbrak}[1]{\ensuremath{{}\left[#1\right.}}
\providecommand{\rsbrak}[1]{\ensuremath{{}\left.#1\right]}}
\providecommand{\brak}[1]{\ensuremath{\left(#1\right)}}
\providecommand{\lbrak}[1]{\ensuremath{\left(#1\right.}}
\providecommand{\rbrak}[1]{\ensuremath{\left.#1\right)}}
\providecommand{\cbrak}[1]{\ensuremath{\left\{#1\right\}}}
\providecommand{\lcbrak}[1]{\ensuremath{\left\{#1\right.}}
\providecommand{\rcbrak}[1]{\ensuremath{\left.#1\right\}}}
\theoremstyle{remark}
\newtheorem{rem}{Remark}
\newtheorem*{remark}{Remark}
\newcommand{\sgn}{\mathop{\mathrm{sgn}}}
\providecommand{\abs}[1]{\vert#1\vert}
\providecommand{\res}[1]{\Res\displaylimits_{#1}} 
\providecommand{\norm}[1]{\lVert#1\rVert}
%\providecommand{\norm}[1]{\lVert#1\rVert}
\providecommand{\mtx}[1]{\mathbf{#1}}
\providecommand{\mean}[1]{E[ #1 ]}
\providecommand{\fourier}{\overset{\mathcal{F}}{ \rightleftharpoons}}
%\providecommand{\hilbert}{\overset{\mathcal{H}}{ \rightleftharpoons}}
\providecommand{\system}{\overset{\mathcal{H}}{ \longleftrightarrow}}
	%\newcommand{\solution}[2]{\textbf{Solution:}{#1}}
\newcommand{\solution}{\noindent \textbf{Solution: }}
\newcommand{\cosec}{\,\text{cosec}\,}
\providecommand{\dec}[2]{\ensuremath{\overset{#1}{\underset{#2}{\gtrless}}}}
\newcommand{\myvec}[1]{\ensuremath{\begin{pmatrix}#1\end{pmatrix}}}
\newcommand{\mydet}[1]{\ensuremath{\begin{vmatrix}#1\end{vmatrix}}}
\numberwithin{equation}{subsection}
\makeatletter
\@addtoreset{figure}{problem}
\makeatother
\let\StandardTheFigure\thefigure
\let\vec\mathbf
\renewcommand{\thefigure}{\theproblem}
\def\putbox#1#2#3{\makebox[0in][l]{\makebox[#1][l]{}\raisebox{\baselineskip}[0in][0in]{\raisebox{#2}[0in][0in]{#3}}}}
     \def\rightbox#1{\makebox[0in][r]{#1}}
     \def\centbox#1{\makebox[0in]{#1}}
     \def\topbox#1{\raisebox{-\baselineskip}[0in][0in]{#1}}
     \def\midbox#1{\raisebox{-0.5\baselineskip}[0in][0in]{#1}}
\vspace{3cm}
\title{Gate Assignment 3}
\author{Yashas Tadikamalla - AI20BTECH11027}
\maketitle
\newpage
\bigskip
\renewcommand{\thefigure}{\theenumi}
\renewcommand{\thetable}{\theenumi}
Download all python codes from 
\begin{lstlisting}
https://github.com/YashasTadikamalla/EE3900/blob/main/GateAssignment3/codes
\end{lstlisting}
%
and latex-tikz codes from 
%
\begin{lstlisting}
https://github.com/YashasTadikamalla/EE3900/blob/main/GateAssignment3/GateAssignment3.tex
\end{lstlisting}
\section{Problem (EC-2005 Q25)}
A linear system is equivalently represented by two sets of state equations:
$$\dot X=AX+BU \text{ and } \dot W=CW+DU$$
Eigenvalues of the representations are also computed as $[\lambda]$ and $[\mu]$. Which of the following is true?
\begin{enumerate}
    \item $[\lambda]=[\mu]$ and $X=W$
    \item $[\lambda]=[\mu]$ and $X\neq W$
    \item $[\lambda]\neq [\mu]$ and $X=W$
    \item $[\lambda]\neq[\mu]$ and $X\neq W$
\end{enumerate}
\section{Solution}
\begin{definition}[State Space representation]
It is a mathematical model of a physical system, as a set of input, output and state variables related by first order difference or differential equations. The most general state representation of a linear system with p inputs, q outputs, and n state variables can be written as
\begin{align}
    \dot X&=AX+BU\\
    Y&=CX+DU
\end{align}
where, $X\in R^n$ is the state vector, $Y\in R^q$ is the output vector, $U\in R^p$ is input vector, $A\in R^{n\times n}$ is the state matrix, $B\in R^{n\times p}$ is input matrix, $C\in R^{q\times n}$ is output matrix, $D\in R^{q\times p}$ is feedthrough matrix.
\end{definition}
\begin{definition}[Eigen values of State Space representation] 
These are the solutions of the charecteristic equation
\begin{align}
    \triangle(\lambda)=det(\lambda I-A)
\end{align}
where A is the state matrix. 
\end{definition}
\begin{theorem}
Consider the n-dimensional continuous time linear system
\begin{align}
    \dot X=AX+BU \text{ and } Y=CX+DU
    \label{eq:org}
\end{align}
Let $T$ be an $n\times n$ real non-singular matrix and let $\bar X= TX$. Then the state equation 
\begin{align}
    \dot{\bar X}=\bar A\bar X+ \bar BU, Y=\bar C\bar X+\bar DU
\end{align}
where $\bar A=TAT^{-1}, \bar B=TB, \bar C=CT^{-1}, \bar D=D$ is said to be equivalent to \eqref{eq:org}.
\label{eq:th1}
\end{theorem}
\begin{proof}
Given, $\dot X=AX+BU \text{ and } Y=CX+DU$, $T$ is a non-singular matrix such that $\bar X= TX$. The same system can be defined using $\bar X$ as the state,
\begin{align}
    \dot{\bar X}&=T\dot X=TAX+TBU\\
    &=TAT^{-1}\bar X+TBU\\
    Y&=CX+DU=CT^{-1}\bar X+DU
\end{align}
\end{proof}
\begin{theorem}
Equivalent state space representations have the same set of eigen values
\label{eq:th2}
\end{theorem}
\begin{proof}
\begin{align}
    \bar\triangle(\lambda)&=det(\lambda I-\bar A)\\
    &=det(\lambda TT^{-1}-TAT^{-1})\\
    &=det(T(\lambda I-A)T^{-1})\\
    &=det(\lambda I-A)=\triangle(\lambda)
\end{align}
\end{proof}
Given,
\begin{align}
    \dot X&=AX+BU \\
    \dot W&=CW+DU
\end{align}
represent the same system. Hence, using \eqref{eq:th1} and \eqref{eq:th2}, we can conclude that
$$[\lambda]=[\mu] \text{ and } X\neq W$$
Hence, option 2 is the correct answer.


Let us now look at a numerical example to establish the correctness of the obtained result. Consider a SISO LTI system of order 2, represented by the equations
\begin{align}
    \dot x_1(t)&=-x_1(t)+1.5x_2(t)+2u(t)\\
    \dot x_2(t)&=4x_1(t)+u(t)\\
    y(t)&=1.5x_1(t)+0.625x_2(t)+u(t)
\end{align}
Its state space representation can be given by \eqref{eq:org}, where
\begin{align}
    X=\begin{bmatrix}
    x_1(t)\\x_2(t)
    \end{bmatrix},Y=y(t)\\
    \dot X=\begin{bmatrix}
    \dot x_1(t)\\\dot x_2(t)
    \end{bmatrix},U=u(t)\\
    A=\begin{bmatrix}
    -1 & 1.5\\
    4 & 0
    \end{bmatrix},B=\begin{bmatrix}
    2\\
    1
    \end{bmatrix}\\
    C=\begin{bmatrix}
    1.5 & 0.625
    \end{bmatrix},D=1
\end{align}
The eigen values for this state representation are
\begin{align}
    det(\lambda I-A)&=0\\
    \begin{vmatrix}
    \lambda+1 & -1.5\\
    -4 & \lambda
    \end{vmatrix}&=0\\
    \lambda^2+\lambda-6&=0\\
    [\lambda]&=\{-3,2\}
\end{align}
Even if we swap the equations, they still should represent the same system. So, consider a different state space representation, 
\begin{align}
    W=\begin{bmatrix}
    x_2(t)\\x_1(t)
    \end{bmatrix},Y=y(t)\\
    \dot W=\begin{bmatrix}
    \dot x_2(t)\\\dot x_1(t)
    \end{bmatrix},U=u(t)
\end{align}
Clearly, $X\neq W$ and $W=TX$, where $T=\begin{bmatrix}
0 & 1\\
1 & 0
\end{bmatrix}$. From \eqref{eq:th1}
\begin{align}
    \bar A=\begin{bmatrix}
    0 & 4\\
    1.5 & -1
    \end{bmatrix},\bar B=\begin{bmatrix}
    1\\
    2
    \end{bmatrix}\\
    \bar C=\begin{bmatrix}
    0.625 & 1.5
    \end{bmatrix},\bar D=1
\end{align}
Also, the eigen values for this state representation are
\begin{align}
    det(\mu I-A)&=0\\
    \begin{vmatrix}
    \mu & -4\\
    -1.5 & \mu+1
    \end{vmatrix}&=0\\
    \mu^2+\mu-6&=0\\
    [\mu]&=\{-3,2\}
\end{align}
Hence, both the state space representations are equivalent, and satisfy $[\lambda]=[\mu] \text{ and } X\neq W$.
\end{document}


