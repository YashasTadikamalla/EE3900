\documentclass{beamer}
\usepackage{listings}
\lstset{
%language=C,
frame=single, 
breaklines=true,
columns=fullflexible
}
\usepackage{gensymb}
\usepackage{subcaption}
\usepackage{url}
\usepackage{tikz}
\usepackage{tkz-euclide} % loads  TikZ and tkz-base
%\usetkzobj{all}
\usetikzlibrary{calc,math}
\usepackage{float}
\newcommand{\myvec}[1]{\ensuremath{\begin{pmatrix}#1\end{pmatrix}}}
\providecommand{\brak}[1]{\ensuremath{\left(#1\right)}}
\newcommand\norm[1]{\left\lVert#1\right\rVert}
\renewcommand{\vec}[1]{\mathbf{#1}}
\providecommand{\abs}[1]{\vert#1\vert}
\usepackage[export]{adjustbox}
\usepackage[utf8]{inputenc}
\usepackage{amsmath}
\newtheorem*{remark}{Remark}
\providecommand{\mbf}{\mathbf}
\providecommand{\pr}[1]{\ensuremath{\Pr\left(#1\right)}}
\providecommand{\qfunc}[1]{\ensuremath{Q\left(#1\right)}}
\providecommand{\sbrak}[1]{\ensuremath{{}\left[#1\right]}}
\providecommand{\lsbrak}[1]{\ensuremath{{}\left[#1\right.}}
\providecommand{\rsbrak}[1]{\ensuremath{{}\left.#1\right]}}
\providecommand{\brak}[1]{\ensuremath{\left(#1\right)}}
\providecommand{\lbrak}[1]{\ensuremath{\left(#1\right.}}
\providecommand{\rbrak}[1]{\ensuremath{\left.#1\right)}}
\providecommand{\cbrak}[1]{\ensuremath{\left\{#1\right\}}}
\providecommand{\lcbrak}[1]{\ensuremath{\left\{#1\right.}}
\providecommand{\rcbrak}[1]{\ensuremath{\left.#1\right\}}}
\usetheme{Boadilla}
\providecommand{\pr}[1]{\ensuremath{\Pr\left(#1\right)}}

\title{Gate Assignment 3 Presentation}
\author{Yashas Tadikamalla}
\date{AI20BTECH11027}
\begin{document}

\begin{frame}
\titlepage
\end{frame}
\begin{frame}
\frametitle{Question}
\begin{block}{Problem (EC-2005 Q25)}
A linear system is equivalently represented by two sets of state equations:
$$\dot X=AX+BU \text{ and } \dot W=CW+DU$$
Eigenvalues of the representations are also computed as $[\lambda]$ and $[\mu]$. Which of the following is true?
\begin{enumerate}
    \item $[\lambda]=[\mu]$ and $X=W$
    \item $[\lambda]=[\mu]$ and $X\neq W$
    \item $[\lambda]\neq [\mu]$ and $X=W$
    \item $[\lambda]\neq[\mu]$ and $X\neq W$
\end{enumerate}
\end{block}
\end{frame}

\begin{frame}
\frametitle{Few prerequisites}
\begin{definition}[State Space representation]
It is a mathematical model of a physical system, as a set of input, output and state variables related by first order difference or differential equations. The most general state representation of a linear system with p inputs, q outputs, and n state variables can be written as
\begin{align}
    \dot X&=AX+BU\\
    Y&=CX+DU
\end{align}
where, $X\in R^n$ is the state vector, $Y\in R^q$ is the output vector, $U\in R^p$ is input vector, $A\in R^{n\times n}$ is the state matrix, $B\in R^{n\times p}$ is input matrix, $C\in R^{q\times n}$ is output matrix, $D\in R^{q\times p}$ is feedthrough matrix.
\end{definition}
\end{frame}

\begin{frame}
\frametitle{}
\begin{definition}[Eigen values of State Space representation] 
These are the solutions of the charecteristic equation
\begin{align}
    \triangle(\lambda)=det(\lambda I-A)=0
\end{align}
where A is the state matrix. 
\end{definition}
\begin{theorem}
Consider the n-dimensional continuous time linear system
\begin{align}
    \dot X=AX+BU, Y=CX+DU
    \label{eq:org}
\end{align}
Let $T$ be $n\times n$ real non-singular matrix and let $\bar X= TX$. Then the state equation 
\begin{align}
    \dot{\bar X}=\bar A\bar X+ \bar BU, Y=\bar C\bar X+\bar DU
    \label{eq:new}
\end{align}
where $\bar A=TAT^{-1}, \bar B=TB, \bar C=CT^{-1}, \bar D=D$ is equivalent to \eqref{eq:org}.
\label{eq:th1}
\end{theorem}
\end{frame}

\begin{frame}
\frametitle{}
\begin{proof}
Given, $\dot X=AX+BU \text{ and } Y=CX+DU$, $T$ is a non-singular matrix such that $\bar X= TX$. The same system can be defined using $\bar X$ as the state,
\begin{align}
    \dot{\bar X}&=T\dot X=TAX+TBU\\
    &=TAT^{-1}\bar X+TBU\\
    Y&=CX+DU=CT^{-1}\bar X+DU
\end{align}
\end{proof}
\begin{theorem}
Equivalent state space representations have same set of eigen values
\label{eq:th2}
\end{theorem}
\end{frame}

\begin{frame}
\frametitle{}
\begin{proof}
For the representation in \eqref{eq:org}, the eigen values [$\lambda$] are such that
\begin{align}
    Ax&=\lambda x\\
    \Rightarrow(A-\lambda I)x&=0\\
    \Rightarrow det(A-\lambda I)&=0
\end{align}
For the representation in \eqref{eq:new}, the eigen values [$\mu$], are such that
\begin{align}
    \bar Ax&=\mu x\\
    \Rightarrow(\bar A-\mu I)x&=0\\
    \Rightarrow(TAT^{-1}-\mu TT^{-1})x&=0\\
    \Rightarrow det(T(A-\mu I)T^{-1})&=0\\
    \Rightarrow det(A-\mu I)&=0
\end{align}
Hence, equivalent state space representations have same set of eigen values.
\end{proof}
\end{frame}

\begin{frame}
\frametitle{Solution}
Given,
\begin{align}
    \dot X&=AX+BU \\
    \dot W&=CW+DU
\end{align}
represent the same system. Hence, using \eqref{eq:th1} and \eqref{eq:th2}, we can conclude that
$$[\lambda]=[\mu] \text{ and } W=TX$$
where T need not be identity matrix.


Hence, option 2 is the correct answer.
\end{frame}

\begin{frame}
\frametitle{Solution Contd.}
Let us now look at a numerical example to establish the correctness of the obtained result. Consider a SISO LTI system of order 2, represented by the equations
\begin{align}
    \dot x_1(t)&=-x_1(t)+1.5x_2(t)+2u(t)\\
    \dot x_2(t)&=4x_1(t)+u(t)\\
    y(t)&=1.5x_1(t)+0.625x_2(t)+u(t)
\end{align}
\end{frame}

\begin{frame}
\frametitle{Solution Contd.}
Its state space representation can be given by \eqref{eq:org}, where
\begin{align}
    X=\begin{bmatrix}
    x_1(t)\\x_2(t)
    \end{bmatrix},Y=y(t)\\
    \dot X=\begin{bmatrix}
    \dot x_1(t)\\\dot x_2(t)
    \end{bmatrix},U=u(t)\\
    A=\begin{bmatrix}
    -1 & 1.5\\
    4 & 0
    \end{bmatrix},B=\begin{bmatrix}
    2\\
    1
    \end{bmatrix}\\
    C=\begin{bmatrix}
    1.5 & 0.625
    \end{bmatrix},D=1
\end{align}
The eigen values for this state representation are
\begin{align}
    det(\lambda I-A)&=0\\
    \begin{vmatrix}
    \lambda+1 & -1.5\\
    -4 & \lambda
    \end{vmatrix}&=0\\
    \lambda^2+\lambda-6&=0\\
    [\lambda]&=\{-3,2\}
\end{align}
\end{frame}

\begin{frame}
\frametitle{Solution Contd.}
Even if we swap the equations, they still should represent the same system. So, consider a different state space representation, 
\begin{align}
    W=\begin{bmatrix}
    x_2(t)\\x_1(t)
    \end{bmatrix},Y=y(t)\\
    \dot W=\begin{bmatrix}
    \dot x_2(t)\\\dot x_1(t)
    \end{bmatrix},U=u(t)
\end{align}
Clearly, $X\neq W$ and $W=TX$, where $T=\begin{bmatrix}
0 & 1\\
1 & 0
\end{bmatrix}$. From \eqref{eq:th1}
\begin{align}
    \bar A=\begin{bmatrix}
    0 & 4\\
    1.5 & -1
    \end{bmatrix},\bar B=\begin{bmatrix}
    1\\
    2
    \end{bmatrix}\\
    \bar C=\begin{bmatrix}
    0.625 & 1.5
    \end{bmatrix},\bar D=1
\end{align}
\end{frame}

\begin{frame}
\frametitle{Solution Contd.}
Also, the eigen values for this state representation are
\begin{align}
    det(\mu I-A)&=0\\
    \begin{vmatrix}
    \mu & -4\\
    -1.5 & \mu+1
    \end{vmatrix}&=0\\
    \mu^2+\mu-6&=0\\
    [\mu]&=\{-3,2\}
\end{align}
Hence, both the state space representations are equivalent, and satisfy $[\lambda]=[\mu] \text{ and } W=TX$, where T need not be identity matrix.
\end{frame}
\end{document}
