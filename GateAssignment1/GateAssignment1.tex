\documentclass[journal,12pt,twocolumn]{IEEEtran}

\usepackage{setspace}
\usepackage{gensymb}
\singlespacing
\usepackage[cmex10]{amsmath}

\usepackage{amsthm}

\usepackage{mathrsfs}
\usepackage{txfonts}
\usepackage{stfloats}
\usepackage{bm}
\usepackage{cite}
\usepackage{cases}
\usepackage{subfig}

\usepackage{longtable}
\usepackage{multirow}

\usepackage{enumitem}
\usepackage{mathtools}
\usepackage{steinmetz}
\usepackage{tikz}
\usepackage{circuitikz}
\usepackage{verbatim}
\usepackage{tfrupee}
\usepackage[breaklinks=true]{hyperref}
\usepackage{graphicx}
\usepackage{tkz-euclide}

\usetikzlibrary{calc,math}
\usepackage{listings}
    \usepackage{color}                                            %%
    \usepackage{array}                                            %%
    \usepackage{longtable}                                        %%
    \usepackage{calc}                                             %%
    \usepackage{multirow}                                         %%
    \usepackage{hhline}                                           %%
    \usepackage{ifthen}                                           %%
    \usepackage{lscape}     
\usepackage{multicol}
\usepackage{chngcntr}

\DeclareMathOperator*{\Res}{Res}
\newtheorem{theorem}{Theorem}[section]
\newtheorem{corollary}{Corollary}[theorem]
\newtheorem{lemma}[theorem]{Lemma}
\newtheorem{definition}{Definition}[section]
\renewcommand\thesection{\arabic{section}}
\renewcommand\thesubsection{\thesection.\arabic{subsection}}
\renewcommand\thesubsubsection{\thesubsection.\arabic{subsubsection}}

\renewcommand\thesectiondis{\arabic{section}}
\renewcommand\thesubsectiondis{\thesectiondis.\arabic{subsection}}
\renewcommand\thesubsubsectiondis{\thesubsectiondis.\arabic{subsubsection}}


\hyphenation{op-tical net-works semi-conduc-tor}
\def\inputGnumericTable{}                                 %%

\lstset{
%language=C,
frame=single, 
breaklines=true,
columns=fullflexible
}
\begin{document}

\newcommand{\BEQA}{\begin{eqnarray}}
\newcommand{\EEQA}{\end{eqnarray}}
\newcommand{\define}{\stackrel{\triangle}{=}}
\bibliographystyle{IEEEtran}
\raggedbottom
\setlength{\parindent}{0pt}
\providecommand{\mbf}{\mathbf}
\providecommand{\pr}[1]{\ensuremath{\Pr\left(#1\right)}}
\providecommand{\qfunc}[1]{\ensuremath{Q\left(#1\right)}}
\providecommand{\sbrak}[1]{\ensuremath{{}\left[#1\right]}}
\providecommand{\lsbrak}[1]{\ensuremath{{}\left[#1\right.}}
\providecommand{\rsbrak}[1]{\ensuremath{{}\left.#1\right]}}
\providecommand{\brak}[1]{\ensuremath{\left(#1\right)}}
\providecommand{\lbrak}[1]{\ensuremath{\left(#1\right.}}
\providecommand{\rbrak}[1]{\ensuremath{\left.#1\right)}}
\providecommand{\cbrak}[1]{\ensuremath{\left\{#1\right\}}}
\providecommand{\lcbrak}[1]{\ensuremath{\left\{#1\right.}}
\providecommand{\rcbrak}[1]{\ensuremath{\left.#1\right\}}}
\theoremstyle{remark}
\newtheorem{rem}{Remark}
\newtheorem*{remark}{Remark}
\newcommand{\sgn}{\mathop{\mathrm{sgn}}}
\providecommand{\abs}[1]{\vert#1\vert}
\providecommand{\res}[1]{\Res\displaylimits_{#1}} 
\providecommand{\norm}[1]{\lVert#1\rVert}
%\providecommand{\norm}[1]{\lVert#1\rVert}
\providecommand{\mtx}[1]{\mathbf{#1}}
\providecommand{\mean}[1]{E[ #1 ]}
\providecommand{\fourier}{\overset{\mathcal{F}}{ \rightleftharpoons}}
%\providecommand{\hilbert}{\overset{\mathcal{H}}{ \rightleftharpoons}}
\providecommand{\system}{\overset{\mathcal{H}}{ \longleftrightarrow}}
	%\newcommand{\solution}[2]{\textbf{Solution:}{#1}}
\newcommand{\solution}{\noindent \textbf{Solution: }}
\newcommand{\cosec}{\,\text{cosec}\,}
\providecommand{\dec}[2]{\ensuremath{\overset{#1}{\underset{#2}{\gtrless}}}}
\newcommand{\myvec}[1]{\ensuremath{\begin{pmatrix}#1\end{pmatrix}}}
\newcommand{\mydet}[1]{\ensuremath{\begin{vmatrix}#1\end{vmatrix}}}
\numberwithin{equation}{subsection}
\makeatletter
\@addtoreset{figure}{problem}
\makeatother
\let\StandardTheFigure\thefigure
\let\vec\mathbf
\renewcommand{\thefigure}{\theproblem}
\def\putbox#1#2#3{\makebox[0in][l]{\makebox[#1][l]{}\raisebox{\baselineskip}[0in][0in]{\raisebox{#2}[0in][0in]{#3}}}}
     \def\rightbox#1{\makebox[0in][r]{#1}}
     \def\centbox#1{\makebox[0in]{#1}}
     \def\topbox#1{\raisebox{-\baselineskip}[0in][0in]{#1}}
     \def\midbox#1{\raisebox{-0.5\baselineskip}[0in][0in]{#1}}
\vspace{3cm}
\title{Gate Assignment 1}
\author{Yashas Tadikamalla - AI20BTECH11027}
\maketitle
\newpage
\bigskip
\renewcommand{\thefigure}{\theenumi}
\renewcommand{\thetable}{\theenumi}
Download all python codes from 
\begin{lstlisting}
https://github.com/YashasTadikamalla/EE3900/blob/main/GateAssignment1/codes
\end{lstlisting}
%
and latex-tikz codes from 
%
\begin{lstlisting}
https://github.com/YashasTadikamalla/EE3900/blob/main/GateAssignment1/GateAssignment1.tex
\end{lstlisting}
\section{Problem (EC-2013 Q8)}
The impulse response of a system is $h(t)=tu(t)$. For an input $u(t-1)$, the output is 
\begin{enumerate}
    \item $\dfrac{t^{2}}{2}u(t)$
    \item $\dfrac{t(t-1)}{2}u(t-1)$
    \item $\dfrac{(t-1)^{2}}{2}u(t-1)$
    \item $\dfrac{t^{2}-1}{2}u(t-1)$
\end{enumerate}
\section{Solution}
\begin{definition}[Laplace Transform]
It is an integral transform that converts a function of a real variable $t$ to a function of a complex variable $s$. The Laplace transform of $f(t)$ is denoted by $\mathcal{L}\cbrak{f(t)}$ or $F(s)$.
\begin{align}
    F(s)=\mathcal{L}\cbrak{f(t)}=\int_{0}^{\infty}e^{-st}f(t)dt
\end{align}
\end{definition}
\begin{remark}
Laplace transform of $f(t)=t^n,n\geq1$ is
\begin{align}
    F(s)=\mathcal{L}\cbrak{t^n}=\dfrac{n!}{s^{n+1}},s>0
    \label{eq:t}
\end{align}
\end{remark}
\begin{proof}
Basis Step: $n=1$
\begin{align}
    \mathcal{L}\cbrak{t}&=\int_{0}^{\infty}e^{-st}tdt\\
    &=\sbrak{\dfrac{te^{-st}}{-s}}_{0}^{\infty}+\dfrac{1}{s}\int_{0}^{\infty}e^{-st}dt\\
    &=0+\sbrak{\dfrac{-1}{s^2}e^{-st}}_{0}^{\infty},s>0\\
    &=\dfrac{1}{s^2},s>0
\end{align}
Inductive Step:
\begin{align}
    \mathcal{L}\cbrak{t^n}&=\int_{0}^{\infty}e^{-st}t^ndt\\
    &=\sbrak{\dfrac{t^{n}e^{-st}}{-s}}_{0}^{\infty}+\dfrac{n}{s}\int_{0}^{\infty}t^{n-1}e^{-st}dt\\
    &=0+\dfrac{n}{s}\mathcal{L}\cbrak{t^{n-1}},s>0\\
    &=\dfrac{n}{s}\mathcal{L}\cbrak{t^{n-1}},s>0\label{eq:e}
\end{align}
To prove that if \eqref{eq:t} holds for $n=k$, it holds for $n=k+1$. From \eqref{eq:e}
\begin{align}
    \mathcal{L}\cbrak{t^{k+1}}&=\dfrac{k+1}{s}\mathcal{L}\cbrak{t^{k}}\\
    &=\dfrac{(k+1)k!}{s(s^{k+1})}=\dfrac{(k+1)!}{s^{k+2}},s>0
\end{align}
By mathematical induction, \eqref{eq:t} is true $\forall n\geq 1$
\end{proof}
\begin{lemma}
For any real number c, 
\begin{align}
    \mathcal{L}\cbrak{u(t-c)}=\dfrac{e^{-cs}}{s}, s>0
    \label{eq:u}
\end{align}
\end{lemma}
\begin{proof}
 \begin{align}
     \mathcal{L}\cbrak{u(t-c)}&=\int_{0}^{\infty}e^{-st}u(t-c)dt=\int_{c}^{\infty}e^{-st}dt\\
     &=\sbrak{-\dfrac{e^{-st}}{s}}_{c}^{\infty}=\dfrac{e^{-cs}}{s}, s>0
 \end{align}
\end{proof}
\begin{definition}[Inverse Laplace Transform]
It is the transformation of a Laplace transform into a function of time. If $F(s)=\mathcal{L}\cbrak{f(t)}$, then the Inverse laplace transform of $F(s)$ is $\mathcal{L}^{-1}\cbrak{F(s)}=f(t)$.
\end{definition}
\begin{lemma}[t-shift rule]
For any real number c,
\begin{align}
    \mathcal{L}\cbrak{u(t-c)f(t-c)}=e^{-cs}F(s)
    \label{eq:uf}
\end{align}
\end{lemma}
\begin{proof}
\begin{align}
    \mathcal{L}\cbrak{u(t-c)f(t-c)}&=\int_{0}^{\infty}e^{-st}u(t-c)f(t-c)dt\\
    &=\int_{c}^{\infty}e^{-st}f(t-c)dt\\
    &=\int_{0}^{\infty}e^{-s(\tau+c)}f(\tau)d\tau \brak{t=\tau+c}\\
    &=e^{-cs}\int_{0}^{\infty}e^{-s\tau}f(\tau)d\tau\\
    &=e^{-cs}F(s)
\end{align}
\end{proof}
\begin{corollary}
\begin{align}
    \mathcal{L}^{-1}\cbrak{e^{-cs}F(s)}=u(t-c)f(t-c)
\end{align}
\end{corollary}
\begin{theorem}[Convolution theorem]
Suppose $F(s)=\mathcal{L}\cbrak{f(t)}, G(s)=\mathcal{L}\cbrak{g(t)}$ exist, then,
\begin{align}
    \mathcal{L}^{-1}\cbrak{F(s)G(s)}=f(t)*g(t)\label{eq:cuf}
\end{align}
\end{theorem}
Given,
\begin{align}
    &h(t)=tu(t)\\
    &x(t)=u(t-1)
\end{align}
To find: $y(t)$. We know, 
\begin{align}
y(t)&=h(t)*x(t)\\
&=\mathcal{L}^{-1}\cbrak{H(s)X(s)}
\label{eq:def}
\end{align}
From \eqref{eq:uf} and \eqref{eq:t}, 
\begin{align}
H(s)=e^{0}\mathcal{L}\cbrak{t}=\dfrac{1}{s^2}
\end{align}
From \eqref{eq:u}, 
\begin{align}
X(s)=\dfrac{e^{-s}}{s}
\end{align}
Substituting in \eqref{eq:def},
\begin{align}
y(t)=\mathcal{L}^{-1}\cbrak{\dfrac{e^{-s}}{s^3}}
% &\therefore y(t)=\begin{cases}
% 	\dfrac{(t-1)^{2}}{2}, & t \geq 1 \\~\\[-1em]
% 	0, & t < 1
% 	\end{cases}\\ 
% &\therefore y(t)=\dfrac{(t-1)^{2}}{2}u(t-1)
\end{align}
Consider 
\begin{align}
    p(t)=\dfrac{t^{2}}{2}
\end{align}
From \eqref{eq:t}
\begin{align}
    P(s)=\dfrac{2!}{2s^3}=\dfrac{1}{s^3}
\end{align}
Further, from \eqref{eq:cuf}, for $c=1$
\begin{align}
    \mathcal{L}^{-1}\cbrak{e^{-s}P(s)}&=u(t-1)p(t-1)\\
    &=u(t-1)\dfrac{(t-1)^2}{2}\\
    \therefore y(t)&=\dfrac{(t-1)^2}{2}u(t-1)
\end{align}
Option 3 is the correct answer.
\begin{align}
    h(t)&=\begin{cases}
	t, & t \geq 0 \\~\\[-1em]
	0, & t <0
	\end{cases}\\
	x(t)&=\begin{cases}
	1, & t \geq 1 \\~\\[-1em]
	0, & t <1
	\end{cases}\\
	y(t)&=\begin{cases}
	\dfrac{(t-1)^2}{2}, & t \geq 1 \\~\\[-1em]
	0, & t <1
	\end{cases}
\end{align}
\begin{figure}[!h]
 \centering
 \includegraphics[width=\columnwidth]{GateAssignment1(1).png}
 \caption{Plot of $x(t)$}
 \label{plot}
\end{figure}

\begin{figure}[!h]
 \centering
 \includegraphics[width=\columnwidth]{GateAssignment1(2).png}
 \caption{Plot of $h(t)$}
 \label{plot}
\end{figure}

\begin{figure}[!h]
 \centering
 \includegraphics[width=\columnwidth]{GateAssignment1(3).png}
 \caption{Plot of $y(t)$}
 \label{plot}
\end{figure}
\end{document}


